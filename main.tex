\documentclass[aspectratio=169]{beamer}

\usepackage[T2A]{fontenc}
\usepackage[utf8]{inputenc}
\usepackage[russian,english]{babel}
\usepackage{graphics}
\usepackage[normalem]{ulem}
\usepackage{tikz} \usetikzlibrary{arrows.meta}
\usepackage{listings}

\graphicspath{{res/}}
\DeclareGraphicsExtensions{.png,.jpg}

\renewcommand{\vec}{\overline}

\usetheme{Copenhagen}

\title{Алгоритмы поиска пути в графе}
\author{Чубий Савва, 9ФМ}
\institute{Лицей НИУ ВШЭ}
\date{}%TODO

\begin{document}
\begin{frame}
    \maketitle
\end{frame}

\begin{frame}\frametitle{Очень краткое введение в оценку ассимптитики}
    $f(x) = O(g(x)) \Leftrightarrow \exists C = const:\ \forall x: \frac{|f(x)|}{|g(x)|} \leq C$
\end{frame}

\begin{frame}\frametitle{Способы хранения графа}
    \begin{columns}[onlytextwidth]
        \begin{column}{0.5\textwidth}
            \centering
            \begin{tikzpicture}
                % Vertex
                \draw (0, 0)  node[left] {1};
                \draw (2, 1)  node[above] {2};
                \draw (3, 2)  node[right] {3};
                \draw (2, -1) node[below] {4};
                % Edge
                \draw[thick]        (2, 1)  -- (2, -1);
                \draw[thick,-Latex] (2, 1)  -- (0, 0);
                \draw[thick,-Latex] (2, -1) -- (0, 0);
                \draw[thick]        (2, 1)  -- (3, 2);
            \end{tikzpicture}
        \end{column}
        \begin{column}{0.5\textwidth}
            \begin{itemize}
                \item Матрица смежности

                    \begin{tabular}{c|cccc}
                          & 1 & 2 & 3 & 4 \\\hline
                        1 & 0 & 0 & 0 & 0 \\
                        2 & 1 & 0 & 1 & 1 \\
                        3 & 0 & 1 & 0 & 0 \\
                        4 & 1 & 1 & 0 & 0 \\
                    \end{tabular}
                \item Список смежности

                    1: ---

                    2: 1 3 4

                    3: 2

                    4: 1 2
                \item Скисок ребер

                    2--1;
                    2--3;
                    2--4;
                    3--2;
                    4--1;
                    4--2
            \end{itemize}
        \end{column}
    \end{columns}
\end{frame}

\begin{frame}\frametitle{Обозначения}
    \begin{itemize}
        \item $G$ --- матрица смежности
        \item $g$ --- список смежности
        \item $E$ --- список ребер
        \item $n = |V|$ --- количество вершин
        \item $m = |E|$ --- количество ребер
    \end{itemize}
\end{frame}

\newcommand{\1}{(1, 0)}
\newcommand{\2}{(0, 1)}
\newcommand{\3}{(1, 3)}
\newcommand{\4}{(2, 2)}
\newcommand{\5}{(3, 3)}
\newcommand{\6}{(3, 1)}

\newcommand{\drawVertexes}{
    \draw \1 node[below] {1} circle(0.1);
    \draw \2 node[left]  {2};
    \draw \3 node[left]  {3};
    \draw \4 node[below] {4};
    \draw \5 node[right] {5} circle(0.1);
    \draw \6 node[right] {6};
}

\newcommand{\drawEdges}{
    \draw[thick] \1 -- \2;
    \draw[thick] \1 -- \6;
    \draw[thick] \2 -- \3;
    \draw[thick] \2 -- \4;
    \draw[thick] \2 -- \6;
    \draw[thick] \3 -- \4;
    \draw[thick] \4 -- \6;
    \draw[thick] \5 -- \6;
}

\newcommand{\drawGraph}{
    \drawVertexes
    \drawEdges
}

\newcommand{\go}[2]{
    \onslide<+->{\draw[very thick,color=red] #1 -- #2;}
}

\newcommand{\Go}[2]{
    \onslide<.->{\draw[very thick,color=red] #1 -- #2;}
}

\begin{frame}[fragile]\frametitle{Depth-first search}
    \begin{columns}[onlytextwidth]
        \begin{column}{0.3\textwidth}
            \centering
            \begin{tikzpicture}
                \drawGraph
                \go{\1}{\6}
                \go{\6}{\2}
                \go{\2}{\3}
                \go{\3}{\4}
                \go{\6}{\5}
            \end{tikzpicture}
        \end{column}
        \begin{column}{0.7\textwidth}
            $O(n + m)$

            \begin{lstlisting}[language=python,mathescape,gobble=16]
                doDfs(g: int[n][], s: int)
                    used = [0] * n

                    dfs(u: int)
                        used[u] = 1
                        for v $\in$ g$_u$
                            dfs(v)

                    dfs(s)

            \end{lstlisting}
        \end{column}
    \end{columns}
\end{frame}

\begin{frame}[fragile]\frametitle{Breadth-first search}
    \begin{columns}[onlytextwidth]
        \begin{column}{0.3\textwidth}
            \centering
            \begin{tikzpicture}
                \drawGraph
                \go{\1}{\2}
                \Go{\1}{\6}
                \go{\2}{\3}
                \Go{\2}{\4}
                \Go{\5}{\6}
            \end{tikzpicture}
        \end{column}
        \begin{column}{0.7\textwidth}
            $O(n + m)$

            \begin{lstlisting}[language=python,mathescape,gobble=16]
                doBfs(g: int[n][], s: int)
                    dist = [-1] * n
                    queue q
                    q.push(s)
                    dist[u] = 0
                    while q $\not=\varnothing$
                        u = q.pop()
                        for v $\in$ g$_u$
                            if dist[v] == -1
                                dist[v] = dist[u] + 1
                                q.push(v)
            \end{lstlisting}
        \end{column}
    \end{columns}
\end{frame}

% TODO Отричательный цикл

\end{document}
